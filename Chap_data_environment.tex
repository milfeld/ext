\pagebreak
\chapter{Data Environment}
\label{chap:data_environment}
The OpenMP \plc{data environment} contains data attributes of variables and
objects.  Many constructs (such as, \code{parallel}, \code{simd}, \code{task}) 
accept clauses to control \plc{data-sharing} attributes
of referenced variables in the construct, where \plc{data-sharing} applies to
whether the attribute of the variable is \plc{shared}, 
is \plc{private} storage, or has special operational characteristics 
(as found in the \code{firstprivate}, \code{lastprivate}, \code{linear}, or \code{reduction} clauses).

The data environment for a device (distinguished as a \plc{device data environment})
is controlled on the host by \plc{data-mapping} attributes, which determine the
relationship of the data on the host, the \plc{original} data, and the data on the
device, the \plc{corresponding} data.
\newline

DATA-SHARING ATTRIBUTES

Data-sharing attributes of variables can be classified as being \plc{predetermined},
\plc{explicitly determined} or \plc{implicitly determined}.

Certain variables and objects have \plc{predetermined} attributes.  
A commonly found case is the loop iteration variable in associated loops 
of a \code{for} or \code{do} construct. It is has a private data-sharing attribute.
Variables with predetermined data-sharing attributes can not be listed in a data-sharing clause; but there are some
exceptions (mainly concerning loop iteration variables).

Variables with explicitly determined data-sharing attributes are those that are
referenced in a given construct and are listed in a data-sharing attribute
clause on the construct. Some of the common data-sharing clauses are:
\code{shared}, \code{private}, \code{firstprivate}, \code{lastprivate}, 
\code{linear}, and \code{reduction}. % Are these all of them?

Variables with implicitly determined data-sharing attributes are those
that are referenced in a given construct, do not have predetermined
data-sharing attributes, and are not listed in a data-sharing
attribute clause of an enclosing construct.
For a complete list of variables and objects with predetermined and
implicitly determined attributes, please refer to the
\plc{Data-sharing Attribute Rules for Variables Referenced in a Construct}
subsection of the OpenMP Specifications document.  
\newline

DATA-MAPPING

The \code{map} clause on a device construct explictly specifies how the list items in
the clause are mapped from the encountering task's data environment (on the host)
to the corresponding item in the device data environment (on the device).
The common \plc{list items} are arrays, array sections, scalars, pointers, and
structure elements (members). 

Procedures and global variables have predetermined data mapping if they appear
within the list or block of a \code{declare target} directive. Also, a C/C++ pointer
is mapped as a zero-length array section, as is a C++ variable that is a reference to a pointer.
% Waiting for response from Eric on this.

Without explict mapping, non-scalar and non-pointer variables within the scope of the \code{target}
construct are implicitly mapped with a \plc{map-type} of \code{tofrom}.
Without explicit mapping, scalar variables within the scope of the \code{target}
construct are not mapped, but have an implicit firstprivate data-sharing
attribute. (That is, the value of the original variable is given to a private
variable of the same name on the device.) This behavior can be changed with
the \code{defaultmap} clause.

The \code{map} clause can appear on \code{target}, \code{target data} and 
\code{target enter/exit data} constructs.  The operations of creation and
removal of device storage as well as assignment of the original list item 
values to the corresponding list items may be complicated when the list 
item appears on multiple constructs or when the host and device storage 
is shared. In these cases the item's reference count, the number of times
it has been referenced (+1 on entry and -1 on exited) in nested (structured)
map regions and/or accumulative (unstructured) mappings, determines the operation.
Details of the \code{map} clause and reference count operation are specified 
in the \plc{map Clause} subsection of the OpenMP Specifications document.
