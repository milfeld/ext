%begin 
\pagebreak
\section{target enter/exit data Constructs}
\label{sec:target_enter_exit_data}
\section{Simple target enter data and target exit data Constructs}
While the "structured data" constructs provide persistent data on a 
device for subsequent \code{target} constructs as shown in the 
\code{target data} examples above, the "unstructure data" constructs 
allow the creation and deletion of data on the device at any appropriate
point within the host code, as shown below with the \code{target enter data} 
and \code{target exit data} constructs.

The following C++ code creates/deletes a vector in a constructor/destructor 
of a class. The constructor creates a vector with \code{target enter data}
and uses an \code{alloc} modifier in the map clause to avoid copying values
to the device. The destructor deletes the data (\code{target exit data})
and uses the \code{delete} modifier in the map clause to avoid copying data
back to the host. Note, the stand-alone \code{target enter data} occurs 
after the host vector is created, and the \code{target exit data}
construct occurs before the host data is deleted.
\cppexample{target_unstructured_data}{1cpp}

The following C code allocates and frees the data member of a Matrix structure.
The \code{init\_matrix} function allocates the memory used in the structure and
uses the \code{target enter data} directive to map it to the target device. The
\code{free\_matrix} function removes the mapped array from the target device
and then frees the memory on the host.  Note, the stand-alone \code{target
enter data} occurs after the host memory is allocated, and the \code{target
exit data} construct occurs before the host data is freed.
\cexample{target_unstructured_data}{1c}

The following Fortran code allocates and deallocates a module array.  The
\code{initialize} subroutine allocates the module array and uses the
\code{target enter data} directive to map it to the target device. The
\code{finalize} subroutine removes the mapped array from the target device and
then deallocates the array on the host.  Note, the stand-alone \code{target
enter data} occurs after the host memory is allocated, and the \code{target
exit data} construct occurs before the host data is deallocated.
\ffreeexample{target_unstructured_data}{1f90}
%end

