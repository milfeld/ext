%\pagebreak
\section{\code{simd} and \code{declare} \code{simd} Constructs}
\label{sec:SIMD}

The following example illustrates the basic use of the \code{simd} construct 
to assure the compiler that the loop can be vectorized.

\cexample{SIMD}{1}

\ffreeexample{SIMD}{1}
 

When a function can be inlined within a loop the compiler has an opportunity to 
vectorize the loop. By guaranteeing SIMD behavior of a function's operations, 
characterizing the arguments of the function and privatizing temporary 
variables of the loop, the compiler can often create faster, vector code for 
the loop. In the examples below the \code{declare} \code{simd} construct is 
used on the \plc{add1} and \plc{add2} functions to enable creation of their 
corresponding SIMD function versions for execution within the associated SIMD 
loop. The functions characterize two different approaches of accessing data 
within the function: by a single variable and as an element in a data array, 
respectively. The \plc{add3} C function uses dereferencing.

The \code{declare} \code{simd} constructs also illustrate the use of 
\code{uniform} and \code{linear} clauses.  The \code{uniform(fact)} clause 
indicates that the variable \plc{fact} is invariant across the SIMD lanes. In 
the \plc{add2} function \plc{a} and \plc{b} are included in the \code{unform} 
list because the C pointer and the Fortran array references are constant.  The 
\plc{i} index used in the \plc{add2} function is included in a \code{linear} 
clause with a constant-linear-step of 1, to guarantee a unity increment of the 
associated loop. In the \code{declare} \code{simd} construct for the \plc{add3} 
C function the  \code{linear(a,b:1)} clause instructs the compiler to generate 
unit-stride loads across the SIMD lanes; otherwise,  costly \emph{gather} 
instructions would be generated for the unknown sequence of access of the 
pointer dereferences.

In the \code{simd} constructs for the loops the \code{private(tmp)} clause is 
necessary to assure that the each vector operation has its own \plc{tmp} 
variable.

\cexample{SIMD}{2}

\ffreeexample{SIMD}{2}


A thread that encounters a SIMD construct executes a vectorized code of the 
iterations. Similar to the concerns of a worksharing loop a loop vectorized 
with a SIMD construct must assure that temporary and reduction variables are 
privatized and declared as reductions with clauses.  The example below 
illustrates the use of \code{private} and \code{reduction} clauses in a SIMD 
construct.

\cexample{SIMD}{3}

\ffreeexample{SIMD}{3}


A \code{safelen(N)} clause in a \code{simd} construct assures the compiler that 
there are no loop-carried dependencies for vectors of size \plc{N} or below. If 
the \code{safelen} clause is not specified, then the default safelen value is 
the number of loop iterations.
 
The \code{safelen(16)} clause in the example below guarantees that the vector 
code is safe for vectors up to and including size 16.  In the loop, \plc{m} can 
be 16 or greater, for correct code execution.  If the value of \plc{m} is less 
than 16, the behavior is undefined.

\cexample{SIMD}{4}

\ffreeexample{SIMD}{4}


The following SIMD construct instructs the compiler to collapse the \plc{i} and 
\plc{j} loops into a single SIMD loop in which SIMD chunks are executed by 
threads of the team. Within the workshared loop chunks of a thread, the SIMD 
chunks are executed in the lanes of the vector units.

\cexample{SIMD}{5}

\ffreeexample{SIMD}{5}


%%% section
\section{\code{inbranch} and \code{notinbranch} Clauses}
\label{sec:SIMD_branch}

The following examples illustrate the use of the \code{declare} \code{simd} 
construct with the \code{inbranch} and \code{notinbranch} clauses. The 
\code{notinbranch} clause informs the compiler that the function \plc{foo} is 
never called conditionally in the SIMD loop of the function \plc{myaddint}. On 
the other hand, the \code{inbranch} clause for the function goo indicates that 
the function is always called conditionally in the SIMD loop inside 
the function \plc{myaddfloat}.

\cexample{SIMD}{6}

\ffreeexample{SIMD}{6}


In the code below, the function \plc{fib()} is called in the main program and 
also recursively called in the function \plc{fib()} within an \code{if} 
condition. The compiler creates a masked vector version and a non-masked vector 
version for the function \plc{fib()} while retaining the original scalar 
version of the \plc{fib()} function.

\cexample{SIMD}{7}

\ffreeexample{SIMD}{7}



%%% section
\section{Loop-Carried Lexical Forward Dependence}
\label{sec:SIMD_forward_dep}


 The following example tests the restriction on an SIMD loop with the loop-carried lexical forward-dependence. This dependence must be preserved for the correct execution of SIMD loops.

A loop can be vectorized even though the iterations are not completely independent when it has loop-carried dependences that are forward lexical dependences, indicated in the code below by the read of \plc{A[j+1]} and the write to \plc{A[j]} in C/C++ code (or \plc{A(j+1)} and \plc{A(j)} in Fortran). That is, the read of \plc{A[j+1]} (or \plc{A(j+1)} in Fortran) before the write to \plc{A[j]} (or \plc{A(j)} in Fortran) ordering must be preserved for each iteration in \plc{j} for valid SIMD code generation.

This test assures that the compiler preserves the loop carried lexical forward-dependence for generating a correct SIMD code.

\cexample{SIMD}{8}

\ffreeexample{SIMD}{8}

