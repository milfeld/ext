\pagebreak
\section{Placement of \code{flush}, \code{barrier}, \code{taskwait} 
and \code{taskyield} Directives}
\label{sec:standalone}

The following example is non-conforming, because the \code{flush}, \code{barrier}, 
\code{taskwait}, and \code{taskyield}  directives are stand-alone directives 
and cannot be the immediate substatement of an \code{if} statement. 

\cexample{standalone}{1c}

The following example is non-conforming, because the \code{flush}, \code{barrier}, 
\code{taskwait}, and \code{taskyield}  directives are stand-alone directives 
and cannot be the action statement of an \code{if} statement or a labeled branch 
target.

\ffreeexample{standalone}{1f90}

The following version of the above example is conforming because the \code{flush}, 
\code{barrier}, \code{taskwait}, and \code{taskyield} directives are enclosed 
in a compound statement. 

\cexample{standalone}{2c}

The following example is conforming because the \code{flush}, \code{barrier}, 
\code{taskwait}, and \code{taskyield} directives are enclosed in an \code{if} 
construct or follow the labeled branch target.

\ffreeexample{standalone}{2f90}


