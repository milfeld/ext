\pagebreak
\section{Device Routines}
\label{sec:device}

\subsection{\code{omp\_is\_initial\_device} Routine}
\label{subsec:device_is_initial}

The following example shows how the \code{omp\_is\_initial\_device} runtime library routine 
can be used to query if a code is executing on the initial host device or on a 
target device. The example then sets the number of threads in the \code{parallel} 
region based on where the code is executing.

\cexample{device}{1}

\ffreeexample{device}{1}

\subsection{\code{omp\_get\_num\_devices} Routine}
\label{subsec:device_num_devices}

The following example shows how the \code{omp\_get\_num\_devices} runtime library routine 
can be used to determine the number of devices.

\cexample{device}{2}

\ffreeexample{device}{2}

\subsection{\code{omp\_set\_default\_device} and \\
\code{omp\_get\_default\_device} Routines}
\label{subsec:device_is_set_get_default}

The following example shows how the \code{omp\_set\_default\_device} and \code{omp\_get\_default\_device} 
runtime library routines can be used to set the default device and determine the 
default device respectively.

\cexample{device}{3}

\ffreeexample{device}{3}

