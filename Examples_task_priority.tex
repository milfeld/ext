\pagebreak
\chapter{Task Priority}
\label{chap:task_priority}



\section{Task Priority}
\label{section:task_priority}

In this example we compute arrays in a matrix through a \plc{compute\_array} routine.
Each task has a priority value equal to the value of the loop variable \plc{i} at the
moment of its creation. A higher priority on a task means that a task is a candidate
to run sooner.

The creation of tasks occurs in ascending order (according to the iteration space of
the loop) but a hint, by means of the \code{priority} clause, is provided to reverse
the execution order.

\cexample{task_priority}{1c}

\fexample{task_priority}{1f}

