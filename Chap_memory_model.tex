\pagebreak
\chapter{Memory Model}
\label{chap:memory_model}

In this chapter, examples illustrate race conditions on access to variables with
shared data-sharing attributes.  A race condition can exist when two
or more threads are involved in accessing a variable in which not all
of the accesses are reads; that is, a WaR, RaW or WaW condition
exists (R=read, a=after, W=write). A RaR does not produce a race condition.
 Ensuring thread execution order at
the processor level is not enough to avoid race conditions, because the
local storage at the processor level (registers, caches, etc.)
must be synchronized so that a consistent view of the variable in the
memory hierarchy can be seen by the threads accessing the variable.

OpenMP provides a shared-memory model which allows all threads access
to \plc{memory} (shared data).  Each thread also has exclusive
access to \plc{threadprivate memory} (private data).  A private
variable referenced in an OpenMP directive's structured block is a
new version of the original variable (with the same name) for each
task (or SIMD lane) within the code block.  The private variable is
initially undefined (except for variables in \code{firstprivate}
and \code{linear} clauses), and the original variable value is
unaltered by assignments to the private variable, (except for
\code{reduction}, \code{lastprivate} and \code{linear} clauses).

Private variables in an outer \code{parallel} region can be
shared by implicit tasks of an inner \code{parallel} region 
(with a \code{share} clause on the inner \code{parallel} directive).
Likewise, a private variable may be shared in the region of an
explicit \code{task} (through a \code{shared} clause).


The \code{flush} directive forces a  consistent view of local variables
of the thread executing the \code{flush}.
When a list is supplied on the directive, only the items (variables)
in the list are guaranteed to be flushed.

The following table lists construct in which implied flushes exist, and the
location of their execution.

%\begin{table}[hb]
\begin{center}
%\caption {Execution Location for Implicit Flushes. } 
\begin{tabular}{ | p{0.6\linewidth} | l | } 
\hline
\code{CONSTRUCT}                                   & \makecell{\code{EXECUTION} \\ \code{LOCATION}} \\
\hline
\code{parallel}                                    & upon entry and exit \\
\hline
\makecell[l]{worksharing \\ \hspace{1.5em}\code{for}, \code{do} 
                         \\ \hspace{1.5em}\code{sections} 
                         \\ \hspace{1.5em}\code{single} 
                         \\ \hspace{1.5em}\code{workshare} }  
                                                   & upon exit \\ 
\hline
\code{critical}                                    & upon entry and exit \\
\hline
\code{target}                                      & upon entry and exit \\
\hline
\code{barrier}                                     & during \\
\hline
\code{atomic} operation with \plc{seq\_cst} clause & upon entry and exit \\
\hline
\code{ordered}*                                    & upon entry and exit \\
\hline
\code{cancel}** and \code{cancellation point}**    & during \\
\hline
\code{target data}                                 & upon entry and exit \\
\hline
\code{target update} + \code{to} clause,   
\code{target enter data}                           & on entry \\
\hline
\code{target update} + \code{from} clause, 
\code{target exit data}                            & on exit \\
\hline
\code{omp\_set\_lock}                              & during \\
\hline
\makecell[l]{ \code{omp\_set/unset\_lock}, \code{omp\_test\_lock}*** 
           \\ \code{omp\_set/unset/test\_nest\_lock}*** }
                                                   & during \\
\hline
task scheduling point                              & \makecell[l]{immediately \\ before and after} \\
\hline
\end{tabular}
%\caption {Execution Location for Implicit Flushes. } 

\end{center}
%\end{table}

* without clauses and with \code{threads} or \code{depend} clauses \newline
** when \plc{cancel-var} ICV is \plc{true} (cancellation is turned on) and cancellation is activated \newline
*** if the region causes the lock to be set or unset

A flush with a list is implied for non-sequentially consistent \code{atomic} operations
(\code{atomic} directive without a \code{seq\_cst} clause), where the list item is the
specific storage location accessed atomically (specified as the \plc{x} variable
in \plc{atomic Construct} subsection of the OpenMP Specifications document).

Examples 1-3 show the difficulty of synchronizing threads through \code{flush} and \code{atomic} directives.
