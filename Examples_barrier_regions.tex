\pagebreak
\section{Binding of \code{barrier} Regions}
\label{sec:barrier_regions}

The binding rules call for a \code{barrier} region to bind to the closest enclosing 
\code{parallel} region. 

In the following example, the call from the main program to \plc{sub2} is conforming 
because the \code{barrier} region (in \plc{sub3}) binds to the \code{parallel} 
region in \plc{sub2}. The call from the main program to \plc{sub1} is conforming 
because the \code{barrier} region binds to the \code{parallel} region in subroutine 
\plc{sub2}.

The call from the main program to \plc{sub3} is conforming because the \code{barrier} 
region binds to the implicit inactive \code{parallel} region enclosing the sequential 
part. Also note that the \code{barrier} region in \plc{sub3} when called from 
\plc{sub2} only synchronizes the team of threads in the enclosing \code{parallel} 
region and not all the threads created in \plc{sub1}.

\cexample{barrier_regions}{1}

\fexample{barrier_regions}{1}


