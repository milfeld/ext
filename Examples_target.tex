\pagebreak
\section{\code{target} Construct}
\label{sec:target}

\subsection{\code{target} Construct on \code{parallel} Construct}
\label{subsec:target_parallel}

This following example shows how the \code{target} construct offloads a code 
region to a target device. The variables \plc{p}, \plc{v1}, \plc{v2}, and \plc{N} are implicitly mapped 
to the target device.

\cexample{target}{1}

\ffreeexample{target}{1}

\subsection{\code{target} Construct with \code{map} Clause}
\label{subsec:target_map}

This following example shows how the \code{target} construct offloads a code 
region to a target device. The variables \plc{p}, \plc{v1} and \plc{v2} are explicitly mapped to the 
target device using the \code{map} clause. The variable \plc{N} is implicitly mapped to 
the target device.

\cexample{target}{2}

\ffreeexample{target}{2}

\subsection{\code{map} Clause with \code{to}/\code{from} map-types}
\label{subsec:target_map_tofrom}

The following example shows how the \code{target} construct offloads a code region 
to a target device. In the \code{map} clause, the \code{to} and \code{from} 
map-types define the mapping between the original (host) data and the target (device) 
data. The \code{to} map-type specifies that the data will only be read on the 
device, and the \code{from} map-type specifies that the data will only be written 
to on the device. By specifying a guaranteed access on the device, data transfers 
can be reduced for the \code{target} region.

The \code{to} map-type indicates that at the start of the \code{target} region 
the variables \plc{v1} and \plc{v2} are initialized with the values of the corresponding variables 
on the host device, and at the end of the \code{target} region the variables 
\plc{v1} and \plc{v2} are not assigned to their corresponding variables on the host device.

The \code{from} map-type indicates that at the start of the \code{target} region 
the variable \plc{p} is not initialized with the value of the corresponding variable 
on the host device, and at the end of the \code{target} region the variable \plc{p} 
is assigned to the corresponding variable on the host device.

\cexample{target}{3}

The \code{to} and \code{from} map-types allow programmers to optimize data 
motion. Since data for the \plc{v} arrays are not returned, and data for the \plc{p} array 
are not transferred to the device, only one-half of the data is moved, compared 
to the default behavior of an implicit mapping.

\ffreeexample{target}{3}

\subsection{\code{map} Clause with Array Sections}
\label{subsec:target_array_section}

The following example shows how the \code{target} construct offloads a code region 
to a target device. In the \code{map} clause, map-types are used to optimize 
the mapping of variables to the target device. Because variables \plc{p}, \plc{v1} and \plc{v2} are 
pointers, array section notation must be used to map the arrays. The notation \code{:N} 
is equivalent to \code{0:N}.

\cexample{target}{4}

In C, the length of the pointed-to array must be specified. In Fortran the extent 
of the array is known and the length need not be specified. A section of the array 
can be specified with the usual Fortran syntax, as shown in the following example. 
The value 1 is assumed for the lower bound for array section \plc{v2(:N)}.

\ffreeexample{target}{4}

A more realistic situation in which an assumed-size array is passed to \code{vec\_mult} 
requires that the length of the arrays be specified, because the compiler does 
not know the size of the storage. A section of the array must be specified with 
the usual Fortran syntax, as shown in the following example. The value 1 is assumed 
for the lower bound for array section \plc{v2(:N)}.

\ffreeexample{target}{4b}

\subsection{\code{target} Construct with \code{if} Clause}
\label{subsec:target_if}

The following example shows how the \code{target} construct offloads a code region 
to a target device.

The \code{if} clause on the \code{target} construct indicates that if the variable 
\plc{N} is smaller than a given threshold, then the \code{target} region will be executed 
by the host device.

The \code{if} clause on the \code{parallel} construct indicates that if the 
variable \plc{N} is smaller than a second threshold then the \code{parallel} region 
is inactive.

\cexample{target}{5}

\ffreeexample{target}{5}

The following example is a modification of the above \plc{target.5} code to show the combined \code{target}
and parallel loop directives. It uses the \plc{directive-name} modifier in multiple \code{if}
clauses to specify the component directive to which it applies. 

The \code{if} clause with the \code{target} modifier applies to the \code{target} component of the 
combined directive, and the \code{if} clause with the \code{parallel} modifier applies 
to the \code{parallel} component of the combined directive.    

\cexample{target}{6}

\ffreeexample{target}{6}
