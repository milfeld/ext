\pagebreak
\chapter{Synchronization}
\label{chap:synchronization}

The \code{barrier} construct is a stand-alone directive that requires all threads
of a team (within a contention group) to execute the barrier and complete
execution of all tasks within the region, before continuing past the barrier.

The \code{critical} construct is a directive that contains a structured block. 
The construct allows only a single thread at a time to execute the structured block (region).
Multiple critical regions may exist in a parallel region, and may
act cooperatively (only one thread at a time in all \code{critical} regions),
or separately (only one thread at a time in each \code{critical} regions when
a unique name is supplied on each \code{critical} construct.
An optional (lock) hint clause may be specified on a named \code{critical} 
construct to provide the OpenMP runtime guidance in selection an locking 
mechanism.

On a finer scale the \code{atomic} construct allows only a single thread at 
a time to have atomic access to a storage location involving a single read, 
write, update or capture statement, and a limited number of combinations 
when specifying the \code{capture} \plc{atomic-clause} clause.  The \plc{atomic-clause}
is required for some expression statements, but are not required for 
\code{update} statements. Please see the details in the \plc{atomic Construct} 
subsection of the \plc{Directives} chapter in the OpenMP Specifications document.

% The following three sentences were stolen from the spec.
The \code{ordered} construct either specifies a structured block in a loop, 
simd, or loop SIMD region that will be executed in the order of the loop 
iterations.  The ordered construct sequentializes and orders the execution 
of ordered regions while allowing code outside the region to run in parallel.

Since OpenMP 4.5 the \code{ordered} construct can also be a stand-alone 
directive that specifies cross-iteration dependences in a doacross loop nest.  
The \code{depend} clause uses a \code{sink} \plc{dependence-type}, along with a 
iteration vector argument (vec) to indicate the iteration that satisfies the 
dependence.  The \code{depend} clause with a \code{source} and 
\plc{dependence-types} specify dependence satisfication. 

The \code{flush} directive is a stand-alone construct that forces a thread's 
temporal local storage (view) of a variable to memory where a consistent view
of the variable storage can be accesses.  When the construct is used without 
a variable list, all the locally thread-visible data as defined by the 
base language are flushed.  A construct with a list applies the flush 
operation only to the items in the list.  The \code{flush} construct also 
effectively insures that no memory (load or store) operation for
the variable set (list items, or default set) may be reordered across 
the \code{flush} directive. 

General-purpose routines provide mutual exclusion semantics through locks, 
represented by lock variables.  
The semantics allows a task to \plc{set}, and hence 
\plc{own} a lock, until it is \plc{unset} by the task that set it. A 
\plc{nestible} lock can be set multiple times by a task, and is used
when in code requires nested control of locks.  A \plc{simple lock} can
only be set once by the owning task. There are specific calls for the two
types of locks, and the variable of a specific lock type cannot be used by the
other lock type.  

Any explicit task will observe the synchronization prescribed in a 
\code{barrier} construct and an implied barrier.  Also, additional synchronizations 
are available for tasks.  All children of a task will wait at a \code{taskwait} (for 
their siblings to complete).  A \code{taskgroup} construct creates a region in which the
current task is suspended at the end of the region until all sibling tasks, 
and their descendants, have completed. 
Scheduling contraints on task execution can be prescribed by the \code{depend}
clause to enforce dependence on previously generated tasks.
More details on controlling task executions can be found in the \plc{Tasking} Chapter
in the OpenMP Specifications document. %(DO REF. RIGHT.)
