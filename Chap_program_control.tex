\pagebreak
\chapter{Program Control}
\label{sec:program_control}

CONDITIONAL COMPILATION and CONTROL   -- PLACEMENT of flush, barrier, taskwait and taskyield

Some specific and elementary concepts of controlling program execution are
illustrated in the examples of this chapter.  Control can be directly
managed with conditional control code (ifdef's with the \code{\_OPENMP} macro, and additional sentinel 
for conditionally compiling in Fortran). The \code{if} clause on some constructs
can direct the runtime to ignore or alter the behavior of the directive.
Of course, base-language \code{if} statements can be used to control the "execution" 
of stand-alone directives (such as \code{flush}, \code{barrier}, \code{taskwait}, and  \code{taskyield}).
However, the directives must appear in a block structure, and not as a substatement as showin in examples 1 and 2.


CANCELLATION

Cancellation (termination) of the normal sequence of execution for the threads in an OpenMP region can
be  accomplished with the \code{cancel} construct.  The construct uses a
\plc{construct-type-clause} to set the region-type to activate for cancellation. 
That is, inclusion  of one of the /plc{construct-type-clause} names \code{parallel}, \code{for}, 
\code{do}, \code{sections} and \code{taskgroup}, on the directive line activates the corresponding regions.  
The \code{cancel} construct is activated by the first encountering thread,  and it
continues execution at the end of the specified region.
The \code{cancel} construct is also a concellation point for any other thread of the team 
to also continue execution at the end of the named region.  

Also, once the specified region has been activated for cancellation any thread that encounnters 
a \code{cancellation point} construct with the same specified region (\plc{construct-type-clause} name
in a specified
region, once the region has been activated for cancellation, continues execution at the end of the region.

For an activated \code{cancel taskgroup} construct, the tasks that
belong to the taskgroup set of the innermost enclosing taskgroup region will be canceled. 

The task that encountered the cancel taskgroup construct continues execution at the end of its
task region. Any task of the taskgroup that has already begun execution will run to completion,
unless it encounters a \code{cancellation point}; tasks that have not begun execution "may" be
discarded as completed tasks.

CONTROL vARIABLES 

  Internal control variable (ICV) are used by implementations to hold values which control the execution
  of OpenMP regions.  Control (and hence the ICVs) may be set as implementation defaults, 
  or set and adjusted through environment variables, clauses, and API functions.  Many of the ICV control
  values are accessible through API function calls.  Also, initial ICV values are reported by the runtime
  once when the \code{OMP\_DISPLAY\_ENV} environment variable is set to \code{TRUE}. 

  As an example, the \plc{nthreads-var} is the ICV that holds the number of threads
  to be used in a \code{parallel} region.  It can be set with the \code{OMP\_NUM\_THREADS} environment variable, 
  the \code{omp\_set\_num\_threads()} API function, or the \code{num\_threads()} clause.  The default \plc{nthreads-var}
  value is implementation defined.  All of the ICVs are presented in the \plc{Internal Control Variables} Section
  of the \plc{Directives} chapter of the OpenMP standard.  Within the same section override relationships 
  and scoping information can be found for applying user specifications and understanding the extent of the control.

NESTED CONSTRUCTS:

Nesting of certain combinations of constructs is permitted, and there are combined/composite constructs
consisting of two or more constructs which can be used when the two (or several) constructs would be used
immediately in succession (closely nested).  A composite construct will have an often obviously modified
restricted clause meaning, relative to when the constructs uncombined. %%[appear separately (singly).

The combined \code{parallel do} and \code{parallel for} constructs are formed by combining the \code{parallel}
construct with one of the loops constructs \code{do} or \code{for}.  The
\code{parallel do SIMD} and \code{parallel for SIMD} constructs are composite constructs (composed from
the parallel loop constructs and the \code{SIMD} construct), because the \code{collapse} clause must
explicitly address the ordering of loop chunking and SIMD "combined" execution.

Certain nestings are forbidden, and often the reasoning is obvious.  Worksharing constructs cannot be nested, and
the \code{barrier} construct cannot be nested inside a worksharing construct, or a \code{critical} construct. 
Also, \code{target} constructs cannot be nested.  

The \code{parallel} constructs can be nested, as well as explict \code{task} constructs.  However, parallel
execution in the nested \code{parallel} construct(s) is control by the \code{OMP\_NESTED} and 
\code{OMP\_MAX\_ACTIVE\_LEVELS} environment variables, and the \code{omp\_set\_nexted()} and 
\code{omp\_set\_max\_active\_levels()} API functions.

More details on nesting can be found in the \plc{Nesting of Regions} of the \plc{Directives} 
Chapter in the OpenMP Specifications document.
