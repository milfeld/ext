\pagebreak
\section{The \code{ordered} Clause and the \code{ordered} Construct}
\label{sec:ordered}

Ordered constructs  are useful for sequentially ordering the output from work that 
is done in parallel. The following program prints out the indices in sequential 
order:

\cexample{ordered}{1}

\fexample{ordered}{1}

It is possible to have multiple \code{ordered} constructs within a loop region 
with the \code{ordered} clause specified. The first example is non-conforming 
because all iterations execute two \code{ordered} regions. An iteration of a 
loop must not execute more than one \code{ordered} region:

\cexample{ordered}{2}

\fexample{ordered}{2}

The following is a conforming example with more than one \code{ordered} construct. 
Each iteration will execute only one \code{ordered} region:

\cexample{ordered}{3}

\fexample{ordered}{3}

