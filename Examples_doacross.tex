\pagebreak
\section{Doacross Loop Nest}
\label{sec:doacross}

An \code{ordered} clause can be used on a loop construct with an integer
parameter argument to define the number of associated loops within 
a \plc{doacross loop nest} where cross-iteration dependences exist.
A \code{depend} clause on an \code{ordered} construct within an ordered 
loop describes the dependences of the \plc{doacross} loops. 

In the code below, the \code{depend(sink:i-1)} clause defines an \plc{i-1} 
to \plc{i} cross-iteration dependence that specifies a wait point for 
the completion of computation from iteration \plc{i-1} before proceeding 
to the subsequent statements. The \code{depend(source)} clause indicates 
when the current iteration satisfies the cross-iteration dependence.
%This example is equivalent to the use of the \code{ordered} 
%construct without a clause in place of 
%the \code{ordered} construct with the \code{depend} clause.
For this example the same sequential ordering could have been achieved 
with an \code{ordered} clause without a parameter on the loop directive, 
and a single \code{ordered} directive without the \code{depend} clause
specified for the statement executing the \plc{bar} function.

\cexample{doacross}{1}

\ffreeexample{doacross}{1}

The following code is similar to the previous example but with 
\plc{doacross loop nest} extended to two nested loops, \plc{i} and \plc{j}, 
as specified by the \code{ordered(2)} clause on the loop directive. 
In the C/C++ code, the \plc{i} and \plc{j} loops are the first and
second associated loops, respectively, whereas
in the Fortran code, the \plc{j} and \plc{i} loops are the first and
second associated loops, respectively.
The \code{depend(sink:i-1,j)} and \code{depend(sink:i,j-1)} clauses in 
the C/C++ code define cross-iteration dependences in two dimensions from 
iterations (\plc{i-1, j}) and (\plc{i, j-1}) to iteration (\plc{i, j}).  
Likewise, the \code{depend(sink:j-1,i)} and \code{depend(sink:j,i-1)} clauses 
in the Fortran code define cross-iteration dependences from iterations 
(\plc{j-1, i}) and (\plc{j, i-1}) to iteration (\plc{j, i}).

\cexample{doacross}{2}

\ffreeexample{doacross}{2}


The following example shows the incorrect use of the \code{ordered} 
directive with a \code{depend} clause.  There are two issues with the code.  
The first issue is a missing \code{ordered}~\code{depend(source)} directive,
which could cause a deadlock.  
The second issue is the \code{depend(sink:i+1,j)} and \code{depend(sink:i,j+1)} 
clauses define dependences on lexicographically later 
source iterations (\plc{i+1, j}) and (\plc{i, j+1}), which could cause 
a deadlock as well since they may not start to execute until the current iteration completes.

\cexample{doacross}{3}

\ffreeexample{doacross}{3}


The following example illustrates the use of the \code{collapse} clause for
a \plc{doacross loop nest}.  The \plc{i} and \plc{j} loops are the associated
loops for the collapsed loop as well as for the \plc{doacross loop nest}.
The example also shows a compliant usage of the dependence source
directive being placed before the corresponding sink directive.
Checking the completion of computation from previous iterations at the sink point can occur after the source statement.

\cexample{doacross}{4}

\ffreeexample{doacross}{4}
