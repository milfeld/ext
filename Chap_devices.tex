\pagebreak
\chapter{Devices}
\label{chap:devices}

The \code{target} construct consists of a \code{target} directive 
and an execution region. The \code{target} region is executed on
the default device or the device specified in the \code{device} 
clause. 

In OpenMP version 4.0, by default, all variables within the lexical
scope of the construct are copied \plc{to} and \plc{from} the
device, unless the device is the host, or the data exists on the
device from a previously executed data-type construct that
has created space on the device and possibly copied host
data to the device storage.

The constructs that explicitly
create storage, transfer data, and free storage on the device
are catagorized as structured and unstructured. The
\code{target} \code{data} construct is structured. It creates
a data region around \code{target} constructs, and is
convenient for providing persistent data throughout multiple
\code{target} regions. The \code{target} \code{enter} \code{data} and 
\code{target} \code{exit} \code{data} constructs are unstructured, because 
they can occur anywhere and do not support a "structure" 
(a region) for enclosing \code{target} constructs, as does the
\code{target} \code{data} construct. 

The \code{map} clause is used on \code{target} 
constructs and the data-type constructs to map host data. It 
specifies the device storage and data movement \code{to} and \code{from}
the device, and controls on the storage duration.

There is an important change in the OpenMP 4.5 specification
that alters the data model for scalar variables and C/C++ pointer variables.
The default behavior for scalar variables and C/C++ pointer variables
in an 4.5 compliant code is \code{firstprivate}. Example
codes that have been updated to reflect this new behavior are
annotated with a description that describes changes required
for correct execution. Often it is a simple matter of mapping
the variable as \code{tofrom} to obtain the intended 4.0 behavior.

In OpenMP version 4.5 the mechanism for target
execution is specified as occuring through a \plc{target task}. 
When the \code{target} construct is encountered a new 
\plc{target task} is generated. The \plc{target task} 
completes after the \code{target} region has executed and all data 
transfers have finished.

This new specification does not affect the execution of 
pre-4.5 code; it is a necessary element for asynchronous 
execution of the \code{target} region when using the new \code{nowait} 
clause introduced in OpenMP 4.5.
