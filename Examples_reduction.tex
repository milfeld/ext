\pagebreak
\section{The \code{reduction} Clause}
\label{sec:reduction}

The following example demonstrates the \code{reduction} clause ; note that some 
reductions can be expressed in the loop in several ways, as shown for the \code{max} 
and \code{min} reductions below:

\cexample{reduction}{1c}

\ffreeexample{reduction}{1f90}

A common implementation of the preceding example is to treat it as if it had been 
written as follows:

\cexample{reduction}{2c}

\fortranspecificstart
\ffreenexample{reduction}{2f90}

The following program is non-conforming because the reduction is on the 
\emph{intrinsic procedure name} \code{MAX} but that name has been redefined to be the variable 
named \code{MAX}.
% blue line floater at top of this page for "Fortran, cont."
\begin{figure}[t!]
\linewitharrows{-1}{dashed}{Fortran (cont.)}{8em}
\end{figure}

\ffreenexample{reduction}{3f90}

The following conforming program performs the reduction using the 
\emph{intrinsic procedure name} \code{MAX} even though the intrinsic \code{MAX} has been renamed 
to \code{REN}.

\ffreenexample{reduction}{4f90}

The following conforming program performs the reduction using 
\plc{intrinsic procedure name} \code{MAX} even though the intrinsic \code{MAX} has been renamed 
to \code{MIN}.

\ffreenexample{reduction}{5f90}
\fortranspecificend

The following example is non-conforming because the initialization (\code{a = 
0}) of the original list item \code{a} is not synchronized with the update of 
\code{a} as a result of the reduction computation in the \code{for} loop. Therefore, 
the example may print an incorrect value for \code{a}.

To avoid this problem, the initialization of the original list item \code{a} 
should complete before any update of \code{a} as a result of the \code{reduction} 
clause. This can be achieved by adding an explicit barrier after the assignment 
\code{a = 0}, or by enclosing the assignment \code{a = 0} in a \code{single} 
directive (which has an implied barrier), or by initializing \code{a} before 
the start of the \code{parallel} region.

\cexample{reduction}{6c}

\fexample{reduction}{6f}

The following example demonstrates the reduction of array \plc{a}.  In C/C++ this is illustrated by the explicit use of an array section \plc{a[0:N]} in the \code{reduction} clause.  The corresponding Fortran example uses array syntax supported in the base language.  As of the OpenMP 4.5 specification the explicit use of array section in the \code{reduction} clause in Fortran is not permitted.  But this oversight will be fixed in the next release of the specification.


\cexample{reduction}{7c}

\ffreeexample{reduction}{7f90}
